\section{Introduction}
People's Postcode Lottery (PPL) is a non-commercial lottery manager operating
throughout Great Britain. The parent company, Novamedia, has charity lotteries
in the Netherlands, Sweden, Norway and Germany. As a lottery operator, PPL are
regulated by the UK Gambling Commission.

PPL have a "Group Tech" department to facilitate their technical requirements.
The department has four main components, listed in descending order of size:

\begin{itemize}
  \item{Development}
  \item{IT}
  \item{Process}
  \item{Project Management}
\end{itemize}

The development department was reorganised in April 2023, and the current
structure is based on the team topologies book. The previous structure was
product teams, comprising the Draw, Payments, Web, Beehive improvement, and
Cloud Engineering teams.

There are two stream-aligned teams (Sales and Retention), two platform teams
(Lottery Core and Cloud Engineering), and an enabling team. The stream-aligned
teams are modelled after the company's two core objectives for the year;
for Retention: maintain an average 1.25\% churn; and for Sales: achieve 16\%
household penetration with 820,000 new tickets sold. Churn is the term for the
percentage of players who cancel their subscription in a given time period, and
household penetration is the percentage of UK households who are playing with
PPL.

\subsection{International alignment}

There has been an increased focus on aligning the sister lotteries in 2023.
Despite having a near-identical model, very little is shared between the UK,
Germany, Sweden, The Netherlands, and Norway. Each have a copy of a bespoke
lottery system called Beehive that was built by external consultants. % WHEN WAS IT BUILT??

The lotteries also all have separate websites and front-end components which
use the same theme that's passed down by Novamedia. This was identified as
an area for improvement, and a new international "Lottery 3.0" team has been
created to address this.

\subsection{Development team projects}

The largest 

\subsection{Development practices}

Agile methodologies:

Most teams work using the Scrum methodology for project management; they work
towards a goal set by their Product Owner (PO) in two-week increments,
inspecting and adapting their approach as they gain new information. Working
this way allows teams to be agile and responsive to change.

Cloud computing:

The department have adopted Cloud Computing through its extensive use of 
Amazon Web Services (AWS). Due to the seasonal nature of the business
(peak times are the holiday season) and the variability of traffic---which
spikes during adverts---the infrastructure as a service (IaaS) model is a good
fit. In particular, the department makes extensive use of "serverless"
computing: a model where the business supplies the back-end logic and the
service-provider manages resource allocation on their behalf; this means the
business only pays for the resources they use.

To manage application infrastructure, PPL use AWS Cloud Development Kit (CDK). CDK
is an Infrastructure as Code (IaC) solution that allows developers to define
their system architecture using the programming language of their choice (PPL
use TypeScript). This solves the problem of inconsistent 

DevOps:

The department uses modern development practices, embracing agile methodologies
(Scrum and Kanban), cloud computing and DevOps. Each team--- Most teams follow the
