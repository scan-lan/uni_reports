\section{Introduction}
People's Postcode Lottery (PPL) is a non-commercial lottery manager operating
throughout Great Britain. The parent company, Novamedia, has charity lotteries
in the Netherlands, Sweden, Norway and Germany. As a lottery operator, PPL are
regulated by the UK Gambling Commission.

PPL have a "Group Tech" department to facilitate their technical requirements.
The department has four main components, listed in descending order of size:

\begin{itemize}
  \item{Development}
  \item{IT}
  \item{Process}
  \item{Project Management}
\end{itemize}

The development department was reorganised in April 2023, and the current
structure is based on the team topologies book. The previous structure was
product teams, comprising the Draw, Payments, Web, Beehive improvement, and
Cloud Engineering teams.

There are two stream-aligned teams (Sales and Retention), two platform teams
(Lottery Core and Cloud Engineering), and an enabling team. The stream-aligned
teams are modelled after the company's two core objectives for the year;
Retention: keep churn to 1.25\%, Sales: achieve xx\% household penetration.

\subsection{International alignment}

There has been an increased focus on aligning the sister lotteries in 2023.
Despite having a near-identical model, very little is shared between the UK,
Germany, Sweden, The Netherlands, and Norway. Each have a copy of a bespoke
lottery system that was built by external consultants. % WHEN WAS IT BUILT??

The lotteries also all have separate websites and front-end components which
use the same theme that's passed down by Novamedia. This was identified as
an area for improvement, and a new international "Lottery 3.0" team has been
created to address this.

\subsection{Development team projects}

The largest 

\subsection{Development practices}

The department uses modern development practices, embracing agile methodologies
(scrum and kanban), cloud computing and devops. Each team---with the exception
of enablement---has a product owner who is responsible for deciding on their
priorities. 
